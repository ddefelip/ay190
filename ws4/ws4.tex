\documentclass[11pt,letterpaper]{article}

% include figures
\usepackage{graphicx}
% get nice colors
\usepackage{xcolor}

% change default font to Palatino (looks nicer!)
\usepackage[latin1]{inputenc}
\usepackage{mathpazo}
\usepackage[T1]{fontenc}
% load some useful math symbols/fonts
\usepackage{latexsym,amsfonts,amsmath,amssymb}

% comfort package to easily set margins
\usepackage[top=1in, bottom=1in, left=1in, right=1in]{geometry}

% spacing after a paragraph
\setlength{\parskip}{.15cm}
% indentation at the top of a new paragraph
\setlength{\parindent}{0.0cm}


\begin{document}

\begin{center}
\Large
Ay190 -- Worksheet 4\\
Daniel DeFelippis\\
Date: \today
\end{center}

%%
%%
%% I worked with Scott Barenfeld
%%
%% All python code can be found in the ws4 directory in my repository
%%
%%

\section{Root Finding}

\subsection*{a}

In this problem, we want to numerically calculate the root of the equation 
$$ E - \omega t - e\sin E = 0 $$ for a given angular frequency $\omega$, 
time $t$, and eccentricity $e$. To do this, we choose the Secant method, 
in which points are recursively chosen via the relation
$$ x_{n+1} = x_n - f(x_n) \frac{x_n - x_{n-1}}{f(x_n) - f(x_{n-1})} $$
until the fractional error of some $x_k$ is below the threshold $\epsilon$, 
which we choose to be $\epsilon = 10^{-10}$. To begin the process, we must
first make two guesses at the root. Since $e$ is small, it is sensible to
choose $x_0 = \omega t$ as the first guess of the root (ignoring the small 
$e\sin x$ term), and then take a point near that, say $x_1 = 0.9x_0$ as the
second guess. After applying the Secant method, we get the results for three different times, shown below.

\begin{center}
    \begin{tabular}{| l | l | l | l |}
    \hline
    time (days) & E & \# of iterations & (x, y) (AU)\\ \hline
    91.0 & 1.58209228899 & 4 & (-0.0112957219731, 0.99965732909) \\ \hline
    182.0 & 3.13096420068 & 3 & (-0.999943518526, 0.0106252886903) \\ \hline
    273.0 & 4.67948910053 & 4 & (-0.0328939450239, -0.999180108689) \\
    \hline
    \end{tabular}
\end{center}

We see that the three times correspond to the Earth being 1/4 along, 2/4
along, and 3/4 along its orbit. The convergence is very quick, taking a
maximum of 4 iterations to be within our $10^{-10}$ threshold.     

\subsection*{b}

Here, we increase the eccentricity drastically, from 0.0167 to 0.99999,
meaning the orbit is now essentially a straight line on the x-axis. The
table of results below now shows the Secant method taking longer to find the 
root, especially for the 1/4 and 3/4 orbit times. 

\begin{center}
    \begin{tabular}{| l | l | l | l |}
    \hline
    time (days) & E & \# of iterations & (x, y) (AU)\\ \hline
    91.0 & 2.30664638749 & 7 & (-0.671217514443, 1.48251347872e-05) \\ \hline
    182.0 & 3.13618964107 & 4 & (-0.999985403763, 1.08059184419e-07) \\ \hline
    273.0 & 4.67948910053 & 6 & (-0.680720102446, -1.46507989671e-05) \\
    \hline
    \end{tabular}
\end{center}

It may be possible, in this case, to decrease the number of iterations 
necessary by using the exact derivative, since we can calculate it 
analytically. Doing so does indeed reduce the number of iterations 
for the 1/4 and 3/4 orbit times by 1. Since $e \approx 1$, ignoring the 
$\sin$ term no longer makes as much sense, so choosing a better first
guess would also accelerate the convergence for this high eccentricity 
orbit calculation.

\end{document}