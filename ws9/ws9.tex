\documentclass[11pt,letterpaper]{article}

% include figures
\usepackage{graphicx}
% get nice colors
\usepackage{xcolor}

% change default font to Palatino (looks nicer!)
\usepackage[latin1]{inputenc}
\usepackage{mathpazo}
\usepackage[T1]{fontenc}
% load some useful math symbols/fonts
\usepackage{latexsym,amsfonts,amsmath,amssymb,wasysym}
% be able to insert code
\usepackage{listings}

% comfort package to easily set margins
\usepackage[top=1in, bottom=1in, left=1in, right=1in]{geometry}

% spacing after a paragraph
\setlength{\parskip}{.15cm}
% indentation at the top of a new paragraph
\setlength{\parindent}{0.0cm}
% make units not slanted in math mode
\newcommand{\unit}[1]{\ensuremath{\, \mathrm{#1}}}

\begin{document}

\begin{center}
\Large
Ay190 -- Worksheet 9\\
Daniel DeFelippis\\
Date: \today
\end{center}

%%
%%
%% I worked with Scott Barenfeld
%%
%% All python code can be found in the ws9 directory in my repository
%%
%%

\section*{Solving Large Systems of Linear Equations}

\section{}

Writing a script to load these data files as matrix objects is fairly easy
using np.loadtxt as shown below.
\lstinputlisting[language=Python, firstline=11, lastline=11]{ws9.py}
Typing "LSE1\_m.shape" gives the dimensions of the matrix LSE1\_m. Doing this,
we see that the matrices LSEi\_m with $i = 1,...,5$ are all square matrices with
number of rows (=number of columns) being 10, 100, 200, 1000, and 2000 respectively.
Using the function "slogdet" located in NumPy's linalg module, we can calculate 
the natural log of the determinant to make sure that the determinant isn't 0 so the 
LSE is solvable. This is indeed true.

\section{}

I wrote my own code that implements the Gauss algorithm. I defined two functions,
one which gets the LSE into the desired triangular form, and the other which 
backsubstitutes to find all of the $x_i$ in the LSE equation $A\textbf{x} = \textbf{b}$.
I ran the algorithm on the contrived example
$$\begin{pmatrix}
2 & 5 & -3 \\
1 & -7 & 4 \\
6 & -2 & 2 
\end{pmatrix}
\begin{pmatrix}
x_1 \\
x_2 \\
x_3 
\end{pmatrix}
=
\begin{pmatrix}
3 \\
-1 \\
4
\end{pmatrix}$$
with the known solution of 
$$ \textbf{x} = \begin{pmatrix}
1 \\
2 \\
3
\end{pmatrix}$$ to see if the algorithm returned the correct answer. It did!

So, I could then use it on the much larger LSE matrices. I timed how long running
the algorithm on each LSE took using the "time" function in the time module to store
the starting and stopping time and then print the difference between the two 
(the times are measured in seconds). The timing results are given for a particular
run of the algorithm on the five matrices. 
\begin{center}
    \begin{tabular}{| l | l | l |}
    \hline
    LSE & Size & Time (seconds) \\ \hline
    1 & (10, 10) & 0.00103902816772  \\ \hline
    2 & (100, 100) & 0.173527002335 \\ \hline
    3 & (200, 200) & 0.333551883698   \\ \hline
    4 & (1000, 1000) & 10.1985230446 \\ \hline
    5 & (2000, 2000) & 50.3575429916 \\
    \hline
    \end{tabular}
\end{center} 
The actual values did sometimes vary, 
probably just due to how many other random processes happened to be running on my
laptop at that time.


\section{}

Next, I try doing solving the same systems with NumPy's "solve" function in its 
linalg module. I get faster results compared to Gaussian Elimination 
as shown in the table below.
\begin{center}
    \begin{tabular}{| l | l | l | l | l |}
    \hline
    LSE & Size & Time (Gauss) & Time (NumPy) & Ratio of Times \\ \hline
    1 & (10, 10) & 0.00103902816772 & 0.000102043151855 & 10.182242990655809 \\ \hline
    2 & (100, 100) & 0.173527002335 & 0.00051212310791 & 338.8384543770586\\ \hline
    3 & (200, 200) & 0.333551883698 & 0.0143749713898 & 23.203655482380803  \\ \hline
    4 & (1000, 1000) & 10.1985230446 & 0.521645069122 & 19.55069384968118\\ \hline
    5 & (2000, 2000) & 50.3575429916 & 3.08289694786 & 16.334487932383144\\
    \hline
    \end{tabular}
\end{center}
NumPy's solver is clearly much better. For all sizes, it is at least an order 
of magnitude faster. 

SciPy also has a bunch of solvers in its "sparse.linalg" module. However, they are
all iterative, and the ones I tried (spsolve and cg) were nowhere near NumPy's solver's
speed. I also tried SciPy's "solve" function located in its own linalg module, and it
performed almost identically well as (but no better than) NumPy's solver.
\end{document}
